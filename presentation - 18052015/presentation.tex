\documentclass[xcolor=table]{beamer}

\usepackage[french]{babel}
\usepackage[latin1]{inputenc}
\usepackage[normalem]{ulem}
\usepackage[T1]{fontenc}
\usepackage{fancyhdr}   %% Pour la gestion des num�ros de page
\usepackage{graphicx}
\usepackage{amsmath}
\usepackage{mathrsfs}
\usepackage{amsfonts}
\usepackage{palatino}        %% Palatino fonts
\usepackage{mathptm}        %% PostScript Type 1 math fonts
\usepackage{dsfont} %% Pour mathds
\usepackage{color}
%%\usepackage{pstricks}
\usepackage{xmpmulti}
\usepackage{hyperref}
\usepackage{multimedia}
\usepackage{multirow}
%\usepackage[table]{xcolor}
\usepackage{fourier-orns}
\usepackage{subfigure}
\usepackage{tikz}

\DeclareMathAlphabet{\mathpzc}{OT1}{pzc}{m}{it}

\definecolor{vert}{rgb}{0.07,0.7,0.00}
\definecolor{gris}{gray}{0.70}
\definecolor{gris2}{gray}{0.95}
\definecolor{bleu}{rgb}{0.19,0.19,0.68}

%table setting
\newcommand\T{\rule{0pt}{2.6ex}}
\newcommand\B{\rule[-1.2ex]{0pt}{0pt}}
\renewcommand{\thesubfigure}{\thefigure.\arabic{subfigure}}

\usetheme{allee_marine} %voir fichier beaerthemeallee_marine.sty   ==> \usetheme{allee_marine}


%%%%%%%%%%%%%%%%%%%%%%%%%% Pr�sentation du document %%%%%%%%%%%%%%%%%%%%%%%%%%
\title[Master 1 Project]{Indexing big colored image bank : Texture 3.0}
\author[Etienne CAILLAUD, Thomas LE BRIS, Ibrahima GUEYE, Ga�tan ADIER]{\textbf{Etienne CAILLAUD, Thomas LE BRIS, Ibrahima GUEYE, Ga�tan ADIER}}
\institute [XLIM-SIC UMR CNRS 7252]{\textbf{XLIM-SIC Laboratory UMR CNRS 7252, Poitiers, France}}
\date{}

%%%%%%%%%%%%%%%%%%%%%%% Num�ro de pages en bas � gauche %%%%%%%%%%%%%%%%%%%%%%
\addtobeamertemplate{footline}{\color{blue}\hfill\insertframenumber/\inserttotalframenumber}

\pgfdeclareimage[height=96mm,width=128mm]{nombidon}{mood_eye_light}
\setbeamertemplate{background}{\pgfuseimage{nombidon}}

\pgfdeclareimage[height=96mm,width=128mm]{nombidon2}{mood_eye_light}
\setbeamertemplate{background}{\pgfuseimage{nombidon2}}

%%----------------------------------------------------------------------------
%% A chaque d�but de sous-section : g�n�re une table des mati�res
%%----------------------------------------------------------------------------
\AtBeginSection[]
{
   \setbeamertemplate{background}{\pgfuseimage{nombidon}}
   \begin{frame}<beamer>
    \frametitle{Outline}
    \tableofcontents[currentsection, hideallsubsections] %% affiche la section courante et les autres en gris�, masque les sous-sections
   \end{frame}
  \setbeamertemplate{background}{\pgfuseimage{nombidon2}}
}

\AtBeginSubsection[]
{
  \setbeamertemplate{background}{\pgfuseimage{nombidon}}
  \begin{frame}<beamer>
    \tableofcontents[sectionstyle=show/shaded,subsectionstyle=show/shaded/hide, subsubsectionstyle =hide]
  \end{frame}
   \setbeamertemplate{background}{\pgfuseimage{nombidon2}}
}

\AtBeginSubsubsection[]
{
  \setbeamertemplate{background}{\pgfuseimage{nombidon}}
  \begin{frame}<beamer>
    \tableofcontents[sectionstyle=show/shaded,subsectionstyle=show/shaded/hide,subsubsectionstyle =show/shaded/hide]
  \end{frame}
   \setbeamertemplate{background}{\pgfuseimage{nombidon2}}
}


%%%%%%%%%%%%%%%%%%%%%%%%%%%%%%%%%%%%%%%%%%%%%%%%%%%%%%%%%%%%%%%%%%%%%%%%%%%%%%
%%%%%%%%%%%%%%%%%%%%%%%%%%%%                       %%%%%%%%%%%%%%%%%%%%%%%%%%%
%%%%%%%%%%%%%%%%%%%%%%%%%%     D�BUT DU DOCUMENT     %%%%%%%%%%%%%%%%%%%%%%%%%
%%%%%%%%%%%%%%%%%%%%%%%%%%%%                       %%%%%%%%%%%%%%%%%%%%%%%%%%%
%%%%%%%%%%%%%%%%%%%%%%%%%%%%%%%%%%%%%%%%%%%%%%%%%%%%%%%%%%%%%%%%%%%%%%%%%%%%%%
\begin{document}
\graphicspath{{images/}}
\setbeamercolor{block title example}{bg = gray}

\begin{frame}
    \vspace{-1.5cm}
    \begin{tikzpicture}[remember picture,overlay]
        \node[xshift=0cm, above=8.6cm] at (current page.south west)
        {\includegraphics[width=40cm,height=0.9cm]{cache_titre.png}};
        \node[xshift=2cm, above=2.8cm] at (current page.south west)
        {\includegraphics[height=1.5cm]{Xlim.png}};
        \node[xshift=11cm, above=3cm] at (current page.south west)
        {\includegraphics[height=1cm]{logo_une.jpg}};
        \node[xshift=6.5cm, above=0.7cm] at (current page.south west)
        {\includegraphics[height=1.6cm]{Lifeclef.png}};
    \end{tikzpicture}
    \titlepage
\end{frame}

%%%%%%%%%%%%%%%%%%%%%%%%%%%%%%%%%%%%%%%%%%%%%%%%%%%%%%%%%%%%%%%%%%%%%%%%%%%%%%%%%%%%%%%%%%%%%%%%%%%%%
%%%%%%%%%%%                        D�but de la pr�sentation                       			 %%%%%%%%
%%%%%%%%%%%%%%%%%%%%%%%%%%%%%%%%%%%%%%%%%%%%%%%%%%%%%%%%%%%%%%%%%%%%%%%%%%%%%%%%%%%%%%%%%%%%%%%%%%%%%
\section{Introduction}
%%-----------------------------------------------------------------------------------------
\begin{frame} \frametitle{Context and environment}

\end{frame}


\section{Team presentation}

\begin{frame} \frametitle{Deadlines}
XLIM-SIC Laboratory of University of Poitiers
\begin{itemize}
\item Noel Richard ( Researcher in Color images): Supervisor
\item David Helbert ( Researcher in Signal-Image-Communications): Supervisor
\item Thierry Urruty ( Researcher in Color images): Customer
\end{itemize}

\end{frame}
%%-----------------------------------------------------------------------------------------

\section{User requirement}
\begin{frame} \frametitle{Software}
\begin{itemize}
 \item Design  software programs:\\
   indexation of  images database,calculate descriptor according to  nature images
\item Adapt the last up to date designed color and texture attributes to the current image classification
\item Compare our results (using CLEF challenge metrics)
\item Provide an abstract of the comparisons and a technical report 
\end{itemize}






%%-----------------------------------------------------------------------------------------
\end{frame}
%%-----------------------------------------------------------------------------------------

\section{Work achievement}
%%-----------------------------------------------------------------------------------------
\begin{frame} \frametitle{SIFT}

\end{frame}
%%-----------------------------------------------------------------------------------------

\begin{frame} \frametitle{C$_2$O}

\end{frame}
%%-----------------------------------------------------------------------------------------

\begin{frame} \frametitle{Classification}

\end{frame}
%%-----------------------------------------------------------------------------------------

\begin{frame} \frametitle{CLEF}

\end{frame}
%%-----------------------------------------------------------------------------------------

\begin{frame} \frametitle{Process flow}

\end{frame}
%%-----------------------------------------------------------------------------------------

\section{Results and Discussion}
\begin{frame} \frametitle{Results}
%%-----------------------------------------------------------------------------------------

\end{frame}
%%-----------------------------------------------------------------------------------------

\begin{frame} \frametitle{Discussion}

\end{frame}
%%-----------------------------------------------------------------------------------------

\section{Project Management}
\begin{frame} \frametitle{SCRUM method}
%%-----------------------------------------------------------------------------------------

\end{frame}
%%-----------------------------------------------------------------------------------------

\section{Conclusion}
\begin{frame} \frametitle{Conclusion}
%%-----------------------------------------------------------------------------------------

\end{frame}
%%-----------------------------------------------------------------------------------------


\section{}
\begin{frame} \frametitle{}
%%-----------------------------------------------------------------------------------------
    \begin{center}
        Thanks for attention
    \end{center}
\end{frame}
%%-----------------------------------------------------------------------------------------


\end{document}


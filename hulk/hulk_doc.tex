\documentclass[12pt]{article}
\usepackage[english]{babel}
\usepackage{amsmath,amsthm}
\usepackage{graphicx}
\usepackage{amsfonts}
\usepackage{indentfirst}
\usepackage{lscape}
\usepackage[top=2.5cm,bottom=2.5cm,right=2.5cm,left=2.5cm]{geometry}
\usepackage{titlesec}
\setcounter{secnumdepth}{5}

% ----------------------------------------------------------------

\begin{document}

\section*{Hulk presentation}


Hulk is a powerful machine destined to the researchers of the XLIM-SIC Laboratory of University of Poitiers. It used for doing big calcul mainly for interactive visualisation of datas via opengl distant(virtualgl).

\subsection*{Hulk Characteristics}
\begin{itemize}
	\item	 46 processors Xeon E5 4650 (in total 368 processors)
	\item	1.427 To of RAM
	\item	20 To of stockage scratch
\end{itemize}
	
The different components are interconnected via a bus Numalink6. Hulk has also two NVIDIA(kepler K20)  cards for the visualisation and GPGPU calcul.

\subsubsection*{Usage and connexion }
The codes must be submitted via an PBS script format.
SFTP and Filezilla are two possilities of connexion for to put or get files on Hulk server. 
SSH is the connexion mode used for doing calcul on the server.
\begin{itemize}
	\item	 Host : hulk.sp2mi.univ-poitiers.fr
	\item	IP : 194.167.50.23 
	\item	Port : 86
\end{itemize}

\subsubsection*{connexion to hulk via SFTP}

The command line allowing to connect on the server is :
 
sftp votre\_login\_de\_compte\_mail\_univ@hulk.sp2mi.univ-poitiers.fr -P 86

cd and ls are used for navigation and moving on the server.

get and put are used for to get or put files  on the server.

\subsubsection*{connexion to hulk via FileZilla}

\begin{itemize}
	\item	 Host : hulk.sp2mi.univ-poitiers.fr
	\item	User ID : votre\_login\_de\_compte\_mail\_univ 
	\item	Port : 86
	\item	Password : which created on Hulk 
	\item	Port : 86
\end{itemize}

\subsubsection*{connexion to hulk via SSH}
The ssh connexion is satarted by using the following command line:
  
ssh votre\_login\_de\_compte\_mail\_univ@hulk.sp2mi.univ-poitiers.fr -P 86

After, for start the computation, all scripts must be submitted via PBS.

\subsubsection*{Where write your datas on Hulk}

All big datas (images, videos, results of calcul...) must be droped in the \lq\lq{}scrastch\rq\rq{} of your principal directory on Hulk.

It\rq{}s necessary to indicate the  patch of this directory in your codes for generating your results files which usually require a big memory space.\\ 

If you need to transfert a file, you must archive your results directory  by the following command line:

tar cvfz mon\_archive.tgz mon\_dossier\_de\_resultats/

You can remove the option \lq\lq{}z\rq\rq{} for to speed up the archive process but it will take up more memory space because the archive is not compressed.

tar cvf mon\_archive.tgz mon\_dossier\_de\_resultats/



 
\end{document}
\documentclass[12pt]{article}
\usepackage[english]{babel}
\usepackage{amsmath,amsthm}
\usepackage{graphicx}
\usepackage{amsfonts}
\usepackage{indentfirst}
\usepackage{lscape}
\usepackage[top=2.5cm,bottom=2.5cm,right=2.5cm,left=2.5cm]{geometry}

% ----------------------------------------------------------------
\begin{document}

\section{State of the art}

Since few years color image processing is a major problem, indeed lot of the colour texture discrimination has been explored in a marginal colour way. The problematic is that today we can do color image recognition on numerical images but we don't have good results on nature images. 

The CLEF contest is an answer to that problematic, the aim of the challenge is to put in competition some university laboratories and company laboratories. In this contest each laboratories could compare its colour texture feature against all the other challengers.

In this way 

\subsection{Key-points}

\subsubsection{Classical}

\subsubsection{Dense Grid}


\subsection{Descriptors}

\subsubsection{SIFT}
Scale-invariant feature transform (or SIFT) is an algorithm in computer vision to detect and describe local features in images. The algorithm was published by David Lowe in 1999.
Applications include object recognition, robotic mapping and navigation, image stitching, 3D modeling, gesture recognition, video tracking, individual identification of wildlife and match moving.

The algorithm is patented in the US; the owner is the University of British Columbia.
\paragraph{}
SIFT keypoints of objects are first extracted from a set of reference images and stored in a database. An object is recognized in a new image by individually comparing each feature from the new image to this database and finding candidate matching features based on Euclidean distance of their feature vectors. From the full set of matches, subsets of keypoints that agree on the object and its location, scale, and orientation in the new image are identified to filter out good matches. The determination of consistent clusters is performed rapidly by using an efficient hash table implementation of the generalized Hough transform. Each cluster of 3 or more features that agree on an object and its pose is then subject to further detailed model verification and subsequently outliers are discarded. Finally the probability that a particular set of features indicates the presence of an object is computed, given the accuracy of fit and number of probable false matches. Object matches that pass all these tests can be identified as correct with high confidence.

\subsubsection{SURF}

\subsubsection{Opponent SIFT}

\subsubsection{C$_2$O}

% ----------------------------------------------------------------
\end{document} 